% Copyright (c) 2015-2017, AIT Austrian Institute of Technology GmbH.
% REM All rights reserved. See file POWERFACTORY_FMU_LICENSE.txt for details.

\chapter{FMI-Compliant Naming Convention}
\label{sec:naming_convention}

An FMI-compliant naming convention is needed that allows to refer to the input variables, output variables and parameters of a \pf model in an unambiguous way (e.g., in the FMI model description).
The \fmipp \pf FMU export utility provides two naming conventions:
\begin{itemize}

  \item \emph{Object parameters}: In \pf, all elements (e.g., loads, lines, transformers ) of an electrical network are represented by objects.
  The values of calculation parameters of theses objects (e.g., power consumption, voltage, loading) can be accessed as output variables.
  Their values can also be directly accessed and changed as inputs for quasi-static steady-state simulations.
  See Section~\ref{sec:naming_convention:obj_param} for more details.

  \item \emph{Simulation events}: In RMS simulations, the object parameters cannot be directly accessed to provide input to the simulated model.
  Instead, a simulation event can be sent to a user-defined \dslmodel, which applies the corresponding changes at run-time.
  See Section~\ref{sec:naming_convention:sim_evt} for more details.
\end{itemize}


\section{Naming convention for object parameters}
\label{sec:naming_convention:obj_param}

Referring to an object parameter is done by concatenating the parameter's name with the associated object's type and name in the following way:
\begin{verbatim}
FMI compliant parameter name = <obj-type>.<obj-name>.<param-name>
\end{verbatim}

For instance, the parameter \texttt{plini} associated to a general load (object type \texttt{ElmLod}) called \texttt{Load} would be referred to as \texttt{ElmLod.Load.plini} in the model description.


\section{Naming convention for simulation events}
\label{sec:naming_convention:sim_evt}

When using \pf in RMS simulation mode, events of type \texttt{EvtParam} can be sent as inputs to user-defined {\dslmodel}s.
Such events can be specified by concatenating a composite frame's slot name with the parameter name defined in the associated \dslmodel (see Section~\ref{sec:export:create_model_rms} for details):
\begin{verbatim}
FMI compliant event name = EvtParam.<slot-name>.<param-name>
\end{verbatim}

For instance, assume that a composite frame (type \texttt{BlkDef}) defines a slot called \texttt{Controller}. When a composite model (type \texttt{ElmComp}) assigns to this slot a user-defined \dslmodel that contains a parameter called \texttt{pext}, an input event that changes this parameter would be referred to as \texttt{EvtParam.Controller.pext} in the model description.


\section{Remarks}

The following restrictions and recommendations apply for objects/slots/parameters defined in \pf models that are associated to input or output variables for the co-simulation interface:
\begin{itemize}
  \item The naming convention above requires to only use \emph{parameter names}, \emph{slot names} and \emph{object names} that \emph{do not contain periods} "." within the \pf model.
  \item The name of objects and slots associated to input variables, output variable and parameters has to be \emph{unique within the model}.
  \item It is recommended to \emph{avoid blanks} in in the names of such objects, and slots.
\end{itemize}
