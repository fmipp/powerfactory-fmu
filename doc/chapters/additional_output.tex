\chapter{Writing results}
\label{sec:additional_output}

Two options to write results from PowerFactory to a file are supported.
The first one is to use the result export utilities provided by PowerFactory itself.
This is convenient as it supports the export of large result files with many result variables at the end of a simulation.
The second option is to leave the output of results to the FMU itself.
This can be useful to write and access results already during simulation runs.
The two options are described in more detail in the following.

\section{PowerFactory results export}
At the end of a simulation or more precisely before the PowerFactory instance is closed, PowerFactory searches for the first results export object \texttt{ComRes} in the active study case and executes it.
This allows the export of a specified result object \texttt{ElmRes} and therefore the export of multiple different variables.
If no results export object is found or writing results failed, a warning is issued.
For further details on how to create a \texttt{ElmRes} or \texttt{ComRes} object please refer to the PowerFactory manual.

\section{Writing additional output}
When using a \pf model within a co-simulation, it is possible to write additional simulation results from \pf that are not specified as FMU outputs.
To do so, simply add text files with file extension \texttt{.info} as additional files, containing a list of variable names in clear text, listing exactly one valid variable name per line (just like the input input/output variable name lists for creating an FMU).
For each of these files, a CSV file with the same name (but file extension \texttt{.csv}) will be generated during a simulation run, containing the simulated values of the variables defined in the \texttt{.info} file.
For instance, adding an additional file called \texttt{extra\_data.info} will at simulation time result in the creation of a file called \texttt{extra\_data.csv}.

\subsection*{Note:}
These \texttt{.info} files have to be specified as additional files when creating an FMU!